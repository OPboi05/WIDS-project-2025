\documentclass{article}
\title{Solution to Assigment(Week 1)}
\author{Harshal Gupta}
\date{\today}

\begin{document}
\maketitle
\section{Q1}
Let A = Probability that it rains today, B = Probability that it rains tomorrow
A and B are independent events.
\begin{enumerate}
    \item P(A) = 0.6
    \item P(B) = 0.5
    \item P($\bar{A} \cap \bar{B}$) = 0.3

    
\end{enumerate}
\begin{math}
 P(A \cup B) = 1-0.3 = 0.7
    => P(A) + P(B) - P(A \cap B) = 0.7 
    => P(A \cap B) = 0.4
\end{math}
\begin{enumerate}
    \item P($A \cup B$) = 0.7
    \item P($A \cap B$) = 0.4
    \item P($A \cap \bar{B}$) = P(A) - P($A \cap B$) = 0.6-0.4 = 0.2 
    \item P($A \Delta B$) = P($A \cup B$) - P($A \cap B$) = 0.7-0.4 = 0.3 
\end{enumerate}

\section{Q2}
set A = {$1,2,3,4,5,6$}
set B = {$1,2,3,4,5,6$}

Sample Set (S) is the cartesian product of the set A and B .
S = A x B
|S| = 6x6 = 36

Favourable cases for X1+X2 = 8:
(1,7) , (2,6), (3,5), (4,4), (5,3), (6,2), (7,1)
Probability of each case = $\frac{1}{6} * \frac{1}{6}$ = $\frac{1}{36}$
Total probability = $7*1/36 = \frac{7}{36}$ 

\section{Q3}
We will apply Bayes Theorem. 

Probability that all are girls = $\frac{1}{2^n}$ 
Probability of having exactly k girls = nCk * $\frac{1}{2^n}$
Probability of choosing 1 girl from k girls = $\frac{k}{n}$

By Bayes Theorem: 
\begin{equation}
    P = \frac{\frac{1}{2^n}}{\Sigma^n_{k=0} k/n * nCk *\frac{1}{2^n}}
\end{equation}
\begin{equation}
    P = \frac{1}{2^{n-1}}
\end{equation}

\section{Q4}
The probability of getting head is p and of getting tail is (1-p)
hence 
$f(x) = pf_d(x) + (1-p)f_c(x)$ \\
Similarly
$F(x) = pF_d(x) + (1-p)F_c(x)$ \\
$E(x) = pE(x_d) + (1-p)E(x_c)$ \\
Var(X) = $p^2Var(x_d) + (1-p)^2Var(x_c) + 2p(1-p)Cov(x_d,x_c)$ \\

\section{Q5}
\begin{math}
    Cov(Z,W) = E(ZW) - E(Z)E(W) \\
    => E((1+X+XY^2)(1+X)) - (E(1+X)*(E(1+X+XY^2)) \\
    => 1+ E(2X) +E(XY^2) + E(X^2Y^2) + E(X^2) - 1 - 2E(X) - (E(X))^2 - E(X)E(XY^2) - E(XY^2) \\
    => Var(X) + E(X^2Y^2) (Since E(X) = 0) \\
    => 1+E(X^2)E(Y^2) \\
    => 1+ 1 = 2. \\
     
    
\end{math}
Var(X) = 1, E(X) = 0 \\
So $E(X^2)$ = 1 similarly $E(Y^2)$ = 1.

\section{Q6}
Probability that he will get atleast one offer = 1-Probability he gets no offer \\
$=>$ P = 1 - $\frac{4^4}{5^4}$ \\
$=>$ P = 0.59  \\
which is approximately 60 percent so he is wrong.

\section{Q7}
1000 is a large sample and each event is independent so we can apply Central Limit theorem. \\
The Binomial distribution here will tend to Normal distribution. \\ 
Mean of this distribution($\mu$) = np = 1000*0.1 = 100 \\
Variance of this distribution($\sigma^2$) = npq = 1000*0.1*0.9 = 90 \\

We want P(X >= 120) \\
$=>$ P(X - 100/$\sqrt{90} >= 20/\sqrt{90})$ \\
 = 1 - $\phi(\frac{20}{\sqrt{90}})$ where $\phi$ is the CDF of normal distribution. \\
 = 1 - 0.982 = 0.018 \\

\section{Q8}
64 is quite a large sample so we can apply Central Limit theorem and approximate it as a Normal dsitribution. Let $X_i$ be the number of sandwiches ith person gets. \\
E($X_i$) = 0*1/4 + 1*1/2 + 2*1/4 = 1 \\
Var($X_i$) = 0*1/4 + 1*1/2 + 4*1/4 = 1.5 \\
Total Mean = 1*64 = 64 \\ 
P($ L<=X <= U$) = 0.95 \\
= P($ L - 64/\sqrt{(1.5)} <= Z <= U - 64/\sqrt{(1.5)} ) = 0.95$  \\ 
Now lower and upper limit turn out to be 1.96 \\
Hence our range = (64 +- 1.96 * sqrt(1.5)) = [61.6, 66.4] \\ 
So number of sandwiches should be in this range for us to be 95 percent sure.

\section{Q9}
\begin{enumerate}
    \item E(X) = 0, E(Y) = 0
    \item Var(X) = 1 = Var(Y)
    \item Cov(X,Y) = $\rho$
\end{enumerate}

\section{Q10}
This is a standard bivariate normal distribution. \\
E(X) = 1, E(Y) = 2, Var(X) = 4, Var(Y) = 3, Cov(X,Y) = 1 \\
Corr(X,Y) = 1/2*root(3) \\
$Z_1$ and $Z_2$ are two independent normal distribution such that $Z_i$ N(0,1) \\
Y = $\sqrt{3}Z_1$ + 2 \\
X = $[\rho Z_1 + \sqrt{1-\rho^2}Z_2]2$ + 1
We can substitute Z1 from Equation 1 and then apply formula of E(X+Y) and Var(X+Y) and we will get \\
E($X|Y=y$) = (y+1)/3 \\
Var($X|Y=y$) = $4 * (1-1/12) = 11/3$  \\

Hence $X|Y=y$ $\sim$ N((y+1)/3 , 11/3)

\section{Q11}
E(Z) = E(3X-2Y) = 3E(X) - 2E(Y) = 0 \\ 
Var(Z) = Var(3X-2Y) = 9Var(X) + 4Var(Y) - 12Cov(X,Y)  = 9 + 16 -12 = 13 \\
Cov(Z,X) = Cov(3X-2Y, X) = 3Cov(X,X) - 2Cov(Y,X) = 3Var(X) - 2 = `1. \\
Corr(Z,X) = Cov(Z,X)/sqrt(Var(Z)*Var(X)) = 1/sqrt(13). \\
Z $\sim$ N(0, 13) since linear combination of bivariate normal is also normal

\end{document}